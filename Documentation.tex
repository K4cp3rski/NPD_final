\documentclass[12pt,a4paper]{article}
\usepackage[english]{babel}
\usepackage[utf8]{inputenc}
\usepackage[T1]{fontenc}
\usepackage{amsmath}
\usepackage{graphicx}
\usepackage[table,xcdraw]{xcolor}
\usepackage{hhline}
\usepackage{hyperref}
\usepackage[margin=0.6in]{geometry}
\usepackage{appendix}
\usepackage{caption}
\usepackage{physics}
\usepackage{verbatim}
\setlength\parindent{0pt}


\title{Documentation of NPD final assignment}
\author{Kacper Cybiński}
\date{\today}

\renewcommand{\phi}{\varphi}

\begin{document}

\maketitle

\section{Installation}

\section{Code Overview}

\subsection{\texttt{download.py}}
\subsubsection{\texttt{get\_sheet\_links\_names}}
This functions is fitted to crawl and return implicit links and names of the \verb|.xlsx| files from the  \href{https://www.gov.pl/web/finanse/udzialy-za-2020-r}{government stat website} used in this program.\\ \\
The function's inner working:
\begin{itemize}
    \item Keyword arguments:
    \begin{itemize}
        \item \verb|year|, default value: 2019. This kwarg is responsible for choosing which year you want to download the files from. It is a simplified application, only restricted to scope of this program, because only sites with data for 2019 and 2020 have so similar urls.
    \end{itemize}
    \item Procedure:
    \begin{itemize}
        \item The procedure crawls webpage and collects all html segments containing href links
        \item Then we cut this list looking for our characterically-shaped file links (36-character coded), as only 5 of those are among the links
        \item Lastly we parse for according names, as we need those for saving the downloaded \verb|.xlsx| files.
    \end{itemize}
\end{itemize}

\end{document}